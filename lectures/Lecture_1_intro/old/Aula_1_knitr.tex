\documentclass{beamer}\usepackage[]{graphicx}\usepackage[]{color}
\definecolor{fgcolor}{rgb}{0.345, 0.345, 0.345}
\newcommand{\hlnum}[1]{\textcolor[rgb]{0.686,0.059,0.569}{#1}}%
\newcommand{\hlstr}[1]{\textcolor[rgb]{0.192,0.494,0.8}{#1}}%
\newcommand{\hlcom}[1]{\textcolor[rgb]{0.678,0.584,0.686}{\textit{#1}}}%
\newcommand{\hlopt}[1]{\textcolor[rgb]{0,0,0}{#1}}%
\newcommand{\hlstd}[1]{\textcolor[rgb]{0.345,0.345,0.345}{#1}}%
\newcommand{\hlkwa}[1]{\textcolor[rgb]{0.161,0.373,0.58}{\textbf{#1}}}%
\newcommand{\hlkwb}[1]{\textcolor[rgb]{0.69,0.353,0.396}{#1}}%
\newcommand{\hlkwc}[1]{\textcolor[rgb]{0.333,0.667,0.333}{#1}}%
\newcommand{\hlkwd}[1]{\textcolor[rgb]{0.737,0.353,0.396}{\textbf{#1}}}%

\usepackage{framed}
\makeatletter
\newenvironment{kframe}{%
 \def\at@end@of@kframe{}%
 \ifinner\ifhmode%
  \def\at@end@of@kframe{\end{minipage}}%
  \begin{minipage}{\columnwidth}%
 \fi\fi%
 \def\FrameCommand##1{\hskip\@totalleftmargin \hskip-\fboxsep
 \colorbox{shadecolor}{##1}\hskip-\fboxsep
     % There is no \\@totalrightmargin, so:
     \hskip-\linewidth \hskip-\@totalleftmargin \hskip\columnwidth}%
 \MakeFramed {\advance\hsize-\width
   \@totalleftmargin\z@ \linewidth\hsize
   \@setminipage}}%
 {\par\unskip\endMakeFramed%
 \at@end@of@kframe}
\makeatother

\definecolor{shadecolor}{rgb}{.97, .97, .97}
\definecolor{messagecolor}{rgb}{0, 0, 0}
\definecolor{warningcolor}{rgb}{1, 0, 1}
\definecolor{errorcolor}{rgb}{1, 0, 0}
\newenvironment{knitrout}{}{} % an empty environment to be redefined in TeX

\usepackage{alltt}
\usefonttheme[onlymath]{serif}

\usepackage[english,portuguese]{babel}
\usepackage{graphicx}
\usepackage{ulem} % Para texto em strikeout
\usepackage{amsmath}
\usepackage{amssymb}
\usepackage{hyperref}

\title{Aula 1: Estatística e Probabilidade}
\subtitle{Análise Quantitativa de Dados Ambientais}
\author{\textbf{Thiago S. F. Silva} - tsfsilva@rc.unesp.br}
\institute{Programa de Pós Graduação em Geografia - IGCE/UNESP}
\date{\today}
\IfFileExists{upquote.sty}{\usepackage{upquote}}{}

\begin{document}




%===============================================================================%
\begin{frame}[plain] % plain avoids a badbox error from page number in title page
  \titlepage
\end{frame}

\begin{frame}{Outline}
  \tableofcontents
\end{frame}
%===============================================================================%


\section{Probabilidade}

%===============================================================================%
\begin{frame}{Probabilidade}

 \begin{itemize}
   \item A base de toda a estatística
    \vfill
   \item Conceitualmente simples\ldots
    \vfill
   \item \ldots mas que rapidamente se torna \textbf{bem complexa}.
    \vfill
   \item A probabilidade mede as ``chances'' de um determinado evento ocorrer
 \end{itemize}
 \vfill
\emph{Ex.: Qual a probabilidade de um inseto ser capturado por uma planta carnívora?}

\end{frame} 
%===============================================================================% 


%===============================================================================
\begin{frame}{Probabilidade}

Para falar de probabilidade, precisamos definir alguns termos:

\begin{itemize}
  \item \textbf{Evento ($A$):} um processo probabilístico (Ex.: $A$ = tentativa de captura) \pause
  \vfill
  \item \textbf{Resultado (\emph{outcome, $A_i$}): }resultado observado do evento (Ex.: $A_1$ = hove captura) \pause
  \vfill
  \item \textbf{Espaço (Universo) Amostral ($S = A_i,...,A_n$):} todos os resultados possíveis de um evento (Ex: $S_{captura} = \{Houve Captura, Não Houve Captura\}$) \pause
  \vfill
    \item Neste exemplo, o espaço amostral é discreto
    
\end{itemize}


\end{frame} 
%===============================================================================%


%===============================================================================
\begin{frame}{Probabilidade em 15 minutos}

\textbf{Axioma 1:}
A probabilidade de qualquer evento dentro do espaço amostral é um número real positivo

\begin{equation*}
    P(A) \in \mathbb{R}, P(A) \geq 0  \quad \forall A \in S
\end{equation*}

\textbf{Axioma 2:}
A soma das probabilidades de todos os resultados dentro do espaço amostral é igual a 1

\begin{equation*}
    \sum_{i=1}^{n}{P(A_i)= 1}
\end{equation*}

\end{frame} 
%===============================================================================%


%===============================================================================
\begin{frame}{Probabilidade em 15 minutos}

\textbf{Regra da Subtração:} A probabilidade de observar um determinado resultado é complementar à probabilidade deste resultado não ser observado


\begin{equation*}
    P(A) = 1 - P(A^c) 
\end{equation*}

\alert{Ex.:} Qual a probabilidade de tirarmos 5 em um dado?


\begin{equation*}
    P(A) = \frac{1}{6} = 1 - P(A^c) = 1 - \frac{5}{6} = \frac{1}{6}
\end{equation*}



\end{frame} 
%===============================================================================%


%===============================================================================
\begin{frame}{Probabilidade em 15 minutos}

\textbf{Regra da Multiplicação:} Se dois eventos são \textbf{independentes}, a probabilidade de que os dois ocorram juntos é o \textbf{produto} da probabilidade de cada evento (\textbf{interseção das probabilidades}, $\cap$)

\begin{equation*}
    P(A \cap B) = P(A) \times P(B)
\end{equation*}

\alert{Ex.:} Qual a probabilidade de tirarmos um 5 \textbf{\emph{e}} um 6 em dois dados?


\begin{equation*}
    P(A \cap B) = P(A) \times P(B) = \frac{1}{6} \times \frac{1}{6} = \frac{1}{36}
\end{equation*}



\end{frame} 
%===============================================================================%


%===============================================================================
\begin{frame}{Probabilidade em 15 minutos}

\textbf{Regra da Adição:} Se dois eventos são \textbf{mutuamente exclusivos (disjuntos)}, a probabilidade de que algum deles ocorra é a \textbf{soma} da probabilidade de cada evento (\textbf{união das probabilidades}, $\cup$)
\begin{equation*}
    P(A \cup B) = P(A) + P(B)
\end{equation*}

\alert{Ex.:} Qual a probabilidade de tirarmos um 5 \textbf{\emph{ou}} um 6 em um dado?


\begin{equation*}
    P(A \cup B) = P(A) + P(B) = \frac{1}{6} + \frac{1}{6} = \frac{2}{6}
\end{equation*}



\end{frame} 
%===============================================================================%

%===============================================================================
\begin{frame}{Probabilidade em 15 minutos}

Se dois eventos \textbf{não} são mutuamente exclusivos, usamos: $P(A \cup B) = P(A) + P(B) - P(A \cap B)$

\alert{Ex.:} Qual a probabilidade de sortearmos um 7 ($A$) \textbf{ou} uma carta de espadas ($B$) de um baralho com 52 cartas?

\begin{equation*}
    P(A=7)= \frac{4}{52} = 0.077, P(B = \text{espadas}) = \frac{13}{52} = 0.25
\end{equation*}

\begin{equation*}
    P(A \cup B) = P(A) + P(B) - P(A \cap B) = 0.077 + 0.25 - (0.077 \times 0.25) = 0.308
\end{equation*}

\center{\textbf{Por que subtrair $P(A \cap B)$?}}

\end{frame} 
%===============================================================================%


%===============================================================================
\begin{frame}{Probabilidade em 15 minutos}

\begin{small}

\textbf{Probabilidade condicional:} é a probabilidade de que um evento ocorra, dado que outro evento relacionado \textbf{já ocorreu}:

$P(A|B) = P(A \cap B)/P(B)$

\alert{Ex.:} Qual a probabilidade de uma carta sorteada ser um 7 ($A$), sabendo que a carta é de espadas ($B$)? 

\begin{equation*}
    P(A=7)= \frac{4}{52} = 0.077, P(B = \text{espadas}) = \frac{13}{52} = 0.25
\end{equation*}

\begin{equation*}
    P(A|B) = \frac{P(A \cap B)}{P(B)} = \frac{0.077 \times 0.25}{0.25} = 0.077
\end{equation*}

\center{\textbf{Se já sabemos que a carta é de espadas, a probabilidade de obter um 7 é 1/13, que equivale a 4/52}}

\end{small}

\end{frame} 
%===============================================================================%



%===============================================================================
\begin{frame}{Probabilidade em 15 minutos}

\textbf{Multiplicação para eventos dependentes:} Se dois eventos são \textbf{dependentes}, a probabilidade de que os dois ocorram juntos pode ser obtida pela relação anterior: 

\begin{equation*}
    P(A|B) = \frac{P(A \cap B)}{P(B)},  \quad P(A \cap B) = P(B) \times P(A|B)
\end{equation*}

\alert{Ex.:} Em Rio Claro, a chance de ser picado por \emph{Aedes egyptii} ($C$) é de 70\% por dia. Assumindo que a chance de um mosquito transmitir o vírus ($T$) é de 50\% , qual a probabilidade de um aluno de estatística pegar dengue hoje?



\end{frame} 
%===============================================================================%


%===============================================================================
\begin{frame}{Probabilidade em 15 minutos}

A probabilida de transmissão é condicional à picada. Se houve picada, $P(A | B) = 0.5$. Se não houve picada, $P(A | B) = 0$.

 
 \begin{equation*}
     P(A \cap B) = P(B) \times P(A | B) 
 \end{equation*}
 
  \begin{equation*}
     P(T \cap C) = P(C) \times P(T | C) = 0.7 \times 0.5 = 0.35
 \end{equation*}
 

\end{frame} 
%===============================================================================%

%===============================================================================
\begin{frame}{Exercício: Gotellli \& Ellison 2$^a$ Ed. Ing. pp. 15-17}

\textbf{Plantas vs. lagartas}

\begin{small}

Em uma paisagem, temos manchas de dois fenótipos de uma planta: $R$ é resistente à herbivoria por lagartas, enquanto $R^c$ não é. Os fenótipos nunca ocorrem juntos na mesma mancha, e fenótipos resistentes ocorrem na paisagem com frequência  de 20$\%$.

A probabilidade de uma mancha ser invadida por lagartas ($C$) é de 0.7, independente da variedade.
\vfill

Assumindo que as lagartas se dispersam igualmente para todas as manchas, e que somente populações resistentes sobrevivem à chegada das lagartas, qual a probabilidade de que uma mancha desapareça devido à herbivoria? 
\vfill
\alert{\textbf{Dica:}} Primeiro calcule as probabilidades de ocorrência das quatro combinações possíveis de resultados.

\end{small}

\end{frame}
%===============================================================================%

%===============================================================================
\begin{frame}{Exercício: Gotellli \& Ellison 2$^a$ Ed. Ing. pp. 15-17}

\textbf{Primeiro Passo:} organizando as informações:
\vfill
Resistente ou suscetível são resultados mutuamente exclusivos:

$P(R) = 0.2, \quad P(R^c) = P(1 - R) = 0.8$
\vfill
Presença e ausência de lagartas são resultados mutuamente exclusivos:

$P(L) = 0.7, \quad P(L^c) = P(1 - L) = 0.3$

\vfill
$S = \{R^cL^c,RL^c,RL,R^cL,RL\}$

\end{frame}
%===============================================================================%


%===============================================================================
\begin{frame}{Exercício: Gotellli \& Ellison 2nd Ed. Ing. pp. 15-17}

\textbf{Segundo Passo:} Expressando as probabilidades. Resistência e invasão por lagartas são eventos independentes:

\vfill
$P(R^cL^c) = P(R^c) \times P(L^c)$
\vfill
$P(RL^c) = P(R) \times P(L^c)$
\vfill
$P(R^cL) = P(R^c) \times P(L)$
\vfill
$P(RL) = P(R) \times P(L)$
\vfill
Multiplicamos porque os eventos são independentes.

\end{frame}
%===============================================================================%

%===============================================================================
\begin{frame}{Exercício: Gotellli \& Ellison 2$^a$ Ed. Ing. pp. 15-17}

\textbf{Terceiro Passo:} Calculando as probabilidades:
\vfill
$P(R^c \cap L^c) = P(R^c) \times P(L^c) = 0.8 \times 0.3 = 0.24$
\vfill
$P(R \cap L^c) = P(R) \times P(L^c) = 0.2 \times 0.3 = 0.06$
\vfill
$P(R^c \cap L) = P(R^c) \times P(L) = 0.8 \times 0.7 = 0.56$
\vfill
$P(R \cap L) = P(R) \times P(L) = 0.2 \times 0.7 = 0.14$


\end{frame}
%===============================================================================%

%===============================================================================
\begin{frame}{Exercício: Gotellli \& Ellison 2$^a$ Ed. Ing. pp. 15-17}
\textbf{Quarto Passo:} Combinando as probabilidades:
\begin{small}
\vfill
$P(R^c \cap L^c) = P(R^c) \times P(L^c) = 0.8 \times 0.3 = 0.24$ : Planta permanece
\vfill
$P(R \cap L^c) = P(R) \times P(L^c) = 0.2 \times 0.3 = 0.06$ : Planta permanece
\vfill
$P(R^c \cap L) = P(R^c) \times P(L) = 0.8 \times 0.7 = 0.56$ : \textbf{Planta desaparece}
\vfill
$P(R \cap L) = P(R) \times P(L) = 0.2 \times 0.7 = 0.14$ : Planta permanece
\vfill
\end{small}
$\mathbf{P(\textbf{Planta desaparece}) = 0.56}$

$P(\text{Planta permanece}) = P((R^c \cap L^c) \cup (R \cap L^c) \cup (R \cap L))  = 0.44$ 

\end{frame}
%===============================================================================%



%===============================================================================
\begin{frame}{Expandindo o Exercício}

\textbf{Plantas vs. lagartas vs. invasoras}

\begin{small}

Mesmo nos fenótipos resistentes, a herbivoria diminui a capacidade competitiva da planta estudada, facilitando o estabelecimento ($I$) de uma espécie invasora. Se há presença da lagarta, a invasão tem uma taxa de sucesso de 60\%, e se não há plantas, o sucesso é garantido (100\%). 
\vfill
Qual a probilidade de que haja invasão, sabendo que as lagartas já atingiram a mancha?

\end{small}

\end{frame}
%===============================================================================%

%===============================================================================
\begin{frame}{Expandindo o Exercício}

\begin{small}

O primeiro impulso é calcular $P(I \cap L) = P(I|L) \times P(L)$. Mas a herbivoria leva à remoção da planta quando esta não é resistente, modificando a probabilidade de invasão.

Temos, assim, duas probabilidades condicionais:

$P(I \cap R^cL) = P(I|R^cL) \times P(R^cL) = 1 \times 0.56 = 0.56$
\vfill
$P(I \cap RL) = P(I|RL) \times P(RL) = 0.6 \times 0.14 = 0.084$
\vfill
$RL$ e $R^cL$ são mutuamente exclusivos, então temos:

$P(I \cap L) = P(I \cap R^cL) \cup P(I \cap RL) = 0.56 + 0.084  = 0.644$


\end{small}

\end{frame}
%===============================================================================%


%===============================================================================
\begin{frame}{Estimando Probabilidades}

\emph{Qual a probabilidade de uma planta carnívora capturar um inseto?}

\begin{itemize}
  \item \textbf{Como podemos estimar essa probabilidade?} \pause
  \vfill
  \item Realizando uma \textbf{contagem} dos sucessos e fracassos da planta, para várias visitas de insetos.\pause
  \vfill
  \item Cada visita individual é uma \textbf{realização} do evento: capturado ou não. Também conhecida como \textbf{réplica} ou \textbf{observação}.\pause
  \vfill
  \item O conjunto de realizações sucessivas compreende um \textbf{experimento}.
\end{itemize}

\end{frame} 
%===============================================================================%


%===============================================================================
\begin{frame}{Frequência vs. Probabilidade}

\begin{centering}

\textbf{Frequência de Captura:}

\begin{equation*}
    F  = \frac{\text{número de capturas}}{\text{número de visitas}}
\end{equation*}


\end{centering}

\end{frame} 
%===============================================================================%


%===============================================================================
\begin{frame}{Frequência vs. Probabilidade}

\begin{centering}

\textbf{Frequência de Captura:}

\begin{equation*}
    F = \frac{\text{número de sucessos}}{\text{número de realizações}}
\end{equation*}


\end{centering}

\end{frame} 
%===============================================================================%


%===============================================================================
\begin{frame}{Frequência vs. Probabilidade}

\begin{centering}

\textbf{Probabilidade de Captura:}

\begin{equation*}
    P(\text{captura}) \approx \lim_{n_t \to \infty} \frac{\text{número de sucessos} (n_r)}{\text{número de realizações } (n_t)}
\end{equation*}

\end{centering}

\end{frame} 
%===============================================================================%

%===============================================================================%
\begin{frame}{Culturas Estatísticas}

\textbf{Estatística Frequentista:}

\begin{itemize}

  \item Associada principalmente a \emph{Sir} Ronald Aymer Fisher, FRS.
\vfill
  \item Se baseia na associação entre frequências e probabilidades.
\vfill
  \item Ex: Jogo uma moeda 100 vezes, obtenho 45 caras e 55 coroas. Estimo minhas probabilidades como 0.45 e 0.55
\vfill
  \item Vê a amostra como uma realização aleatória de um evento
\vfill
  \item Parte do princípio de que se o processo fosse repetido infinitamente, seria possivel estimar as probabilidades associadas aos resultados do evento
  
\end{itemize}
 
\end{frame}
%===============================================================================%


%===============================================================================%
\begin{frame}{Culturas Estatísticas}

\begin{centering}

  $p < 0.05$? \pause
\vfill
  $P(A|H)$ ou $P(H|A)$? \pause
\vfill
  É a mesma coisa?

\end{centering}
 
\end{frame}
%===============================================================================%



%===============================================================================%
\begin{frame}{Culturas Estatísticas}

\begin{small}

Na visão frequentista, se avalia a probabilidade de se obter a amostra observada, dada uma determinada hipótese:

\begin{equation*}
P(A|H)
\end{equation*}

\alert{Ex:} Joguei uma moeda 100 vezes, e obtive 65 caras e 35 coroas. Se a minha hipótese é de que a moeda é honesta ($P(\text{cara}) = P(\text{coroa}) = 0.5$), qual chance de eu obter esse resultado, \textbf{ou um resultado mais extremo}?

\begin{equation*}
p =\ensuremath{8.6\times 10^{-4}}
\end{equation*}

Se eu repetir esse experimento infinitas vezes (jogar 100 moedas), vou encontrar um resultado igual ou mais extremo 0.086\% das vezes.

\end{small} 
 
\end{frame}
%===============================================================================%

%===============================================================================%
\begin{frame}{Culturas Estatísticas}

A lógica nos diz que o mais importante é saber $P(H|S)$. Mas como? \pause

\centering{\textbf{Teorema de Bayes:}}

\begin{equation*}
P(H \mid A) = \frac{P(H \cap A)}{P(A)} =\frac{P(A \mid H) \times P(H)}{P(A)}
\end{equation*}

$P(A | H)$: probabilidade da amostra se a hipótese é verdadeira

$P(A)$: probabilidade da amostra, garante que $0 \leq P(H | A) \leq 1$

$\mathbf{P(H)}$: probabilidade da hipótese ser verdadeira. Conhecida como \textbf{priori (prior)}.



\end{frame}
%===============================================================================%

%===============================================================================%
\begin{frame}{Culturas Estatísticas}

\textbf{Estatística Bayesiana}

\begin{itemize}

  \item Associada a Thomas Bayes

 \item Na visão bayesiana, a análise estatística serve para \textbf{atualizar} o conhecimento anterior

 \item O conhecimento prévio pode ser usado para definir uma probabilidade \emph{priori} da hipótese ser verdadeira 
 
 \item O resultado do experimento permite que voce atualize (melhore) essa estimativa de probabilidade, com base na amostra observada.

\end{itemize}

\end{frame}
%===============================================================================%


%===============================================================================%
\begin{frame}{Estatística Bayesiana}

\alert{Ex.:} Joguei uma moeda 100 vezes, e obtive 65 caras e 35 coroas. Se a minha hipótese é de que a moeda é honesta ($P(\text{cara}) = P(\text{coroa}) = 0.5$), qual a probabilidade que essa hipótese esteja correta?
\vfill

$H_0$: A moeda é honesta

$H_1$: A moeda é tendenciosa

\vfill
Baseado em meu conhecimento de moedas, eu poderia dizer que a probabilidade dela ser honesta é 0.9 ($P(H_0) = 0.9$), e a probabilidade dela ser tendenciosa é 0.1 ($P(H_1) = 0.1$). 
 
\end{frame}
%===============================================================================%

%===============================================================================%
\begin{frame}{Estatística Bayesiana}

Para H$_0$:

\begin{equation*}
P(H_0|A) = \frac{P(A|H_0) \times P(H_0)}{P(A)}
\end{equation*}

\begin{equation*}
P(H_0 | A) = \frac{P(A | H_0) \times P(H_0)}{P(A | H_0) \times P(H_0) + P(A | H_1) \times P(H_1)}
\end{equation*}

\begin{equation*}
P(H_0 | A) = \frac{P(\ensuremath{8.6\times 10^{-4}})P(0.9)}{\ensuremath{8.6\times 10^{-4}}\times 0.9 +0.0834 \times 0.1}
\end{equation*}

$P(H_0|A) = 0.085$
 
\end{frame}
%===============================================================================%

%===============================================================================%
\begin{frame}{Estatística Bayesiana}

Para H$_1$:

\begin{equation*}
P(H_1 | A) = \frac{P(A | H_1) \times P(H_1)}{P(A)}
\end{equation*}

\begin{equation*}
P(H_1 | A) = \frac{P(A | H_1) \times P(H_1)}{P(A | H_1) \times P(H_1) + P(A | H_0) \times P(H_0)}
\end{equation*}

\begin{equation*}
P(H_0 | A) = \frac{P(0.0834) \times P(0.1)}{0.0834 \times 0.1 +\ensuremath{8.6\times 10^{-4}} \times 0.9}
\end{equation*}

$P(H_1|A) = 0.915$
 
\end{frame}
%===============================================================================%



%===============================================================================%
\begin{frame}{Estatística Bayesiana}

A escolha da \emph{priori} afeta fortememente a \emph{posteriori}:
\vfill

\begin{small}

$\mathbf{P(H_0=0.5),P(H_1=0.5)} \rightarrow P(H_0|S) = 0.01, P(H_1|S) = 0.98$

\vfill

$\mathbf{P(H_0=0.75),P(H_1=0.25)} \rightarrow P(H_0|S) = 0.03 , P(H_1|S) = 0.97$

\vfill

$\mathbf{P(H_0=0.95),P(H_1=0.05)} \rightarrow P(H_0|S) = 0.16, P(H_1|S) = 0.84$

\vfill

$\mathbf{P(H_0=0.99),P(H_1=0.01)} \rightarrow P(H_0S) = 0.506, P(H_1|S) = 0.494$

\end{small}

\end{frame}
%===============================================================================%

%===============================================================================%
\begin{frame}{Frequentista vs. Bayesiana}

\begin{columns}[lr]

\column{.5\textwidth}

\begin{scriptsize}
\textbf{Bayesianos sobre frequentistas:
}
\begin{itemize}

  \item{Ignoram qualquer informação a priori}
  
  \item{Se baseiam em experimentos fictícios}
  
\end{itemize}

\vspace{.5in}

\textbf{Frequentistas sobre Bayesianos:}

\begin{itemize}

  \item{Podem gerar o resultado que quiserem manipulando as \emph{priori}}
  
\end{itemize}

\end{scriptsize}

\column{.5\textwidth}

\end{columns}


\end{frame}
%===============================================================================%


%===============================================================================%
\begin{frame}{Qual vamos usar?}

\textbf{O Curso se baseará na filosofia frequentista.}

\begin{itemize}

\item Mais frequentemente usada (\emph{tu-dum psh}).

\item Mais familiar à comunidade ecológico-científica.

\item É a estatística com a qual voces já tiveram algum contato prévio.

\item É a que eu sei ensinar.

\end{itemize}

\textbf{Contudo, tomaremos cuidado em enfatizar os maus usos e compreensões equivocadas da estatística frequentista.}
  
  
\end{frame}
%===============================================================================%


%===============================================================================%
%===============================================================================%
\end{document}
%===============================================================================%
%===============================================================================%
