\documentclass{beamer}
\usefonttheme[onlymath]{serif}

\usepackage[portuguese,english]{babel}
\usepackage{graphicx}
\usepackage{ulem} % Para texto em strikeout
\usepackage{amsmath}
\usepackage{amssymb}

\ifdefined\knitrout 
\renewenvironment{knitrout}{\setlength{\topsep}{0mm}}{}
\else
\fi

\title{Aulas 2-4: Distribuições de Probabilidade e Testes de Hipótese}
\subtitle{Análise Quantitativa de Dados Ambientais}
\author{\textbf{Thiago S. F. Silva} - tsfsilva@rc.unesp.br}
\institute{Programa de Pós Graduação em Geografia - IGCE/UNESP}
\date{\today}

\usepackage{Sweave}
\begin{document}
\Sconcordance{concordance:Aula_2_only_cont.tex:Aula_2_only_cont.Rnw:%
1 20 1 1 0 2 1 1 6 60 1 1 2 7 0 1 2 130 1 1 20 8 1 1 3 2 0 3 1 1 5 3 0 %
1 2 9 0 1 3 7 1 1 15 7 1 1 3 2 0 3 1 1 4 2 0 1 2 9 0 1 3 7 1 1 15 7 1 1 %
4 12 0 1 3 7 1 1 15 27 1}



%===============================================================================%
\begin{frame}[plain] % plain avoids a badbox error from page number in title page
  \titlepage
\end{frame}

\begin{frame}{Outline}
  \tableofcontents
\end{frame}
%===============================================================================%




%===============================================================================
\begin{frame}{Distribuição de Poisson}

\small{\textbf{Poisson}: A Binomial modela o número de sucessos esperados com um numero fixo de realizações. A distribuição Poisson modela a ocorrência de sucessos em situações onde o número de realizações é infinito. O exemplo mais comum são dados de contagem de indivíduos em parcelas, ou ao longo de um intervalo de tempo}. 

\vfill

\textbf{Distribuição Poisson:} $X \sim Pois(\lambda)$ 
  
Parâmetros: $\lambda$ ($\lambda > 0$)

Suporte: $x = k, k \in \left\{0,\ldots,n\right\}$

p.m.f:
  \begin{equation*}
    f(x) = \frac{lambda^k}{k!} e^{-\lambda}
  \end{equation*}

$E[X] = \lambda$

$Var[X] = \lambda$

\end{frame} 
%===============================================================================%

%===============================================================================
\begin{frame}[fragile]


Ex.: Se em média eu observo 50 indivíduos por parcela, qual a probabilidade de eu observar uma parcela com 100 indivíduos?

\begin{equation*}
  f(x) = \frac{lambda^k}{k!} e^{-\lambda}
\end{equation*}

\begin{equation*}
  P(x=100) = \frac{50^{100}}{100!} \times e^{-50}
\end{equation*}

\begin{equation*}
  P(x=100) = 845272575844 \times  1.92875 \times 10^{-22}
\end{equation*}

\begin{equation*}
  P(x=100) = 1.630319 \times 10^{-10}
\end{equation*}

\begin{Schunk}
\begin{Sinput}
> dpois(100,50)
\end{Sinput}
\begin{Soutput}
[1] 1.630319e-10
\end{Soutput}
\end{Schunk}



\end{frame} 
%===============================================================================%


%===============================================================================
\begin{frame}{Distribuições Contínuas}

\begin{itemize}
  \item Até agora falamos de V.A. discretas
  \item Mas e se os dados que queremos modelar são contínuos?
  \item \textbf{Exemplo:} Qual a probabilidade da temperatura máxima de hoje ser 32$^\circ$C?
\end{itemize}

\end{frame} 
%===============================================================================%

%===============================================================================
\begin{frame}{Distribuições Contínuas}

\begin{itemize}
  \item Uma V.A. contínua pode assumir infinitos valores
  
  \vfill
  
  \item Se assumimos que: $P \approx F = \frac{n_i}{N}$
  
  \vfill
  
  \item Qual dos dois resultados tem probabilidade maior?
  
  \end{itemize}

\centering
  P(temperatura máxima de hoje) =  32$^\circ$C?
  
  ou
  
  P(temperatura máxima de hoje) = 32.354321$^\circ$C?
  
\end{frame} 
%===============================================================================%


%===============================================================================
\begin{frame}{Distribuições Contínuas}

Qual dos dois resultados tem probabilidade maior?
  \vfill
\centering

P(temperatura máxima de hoje) =  32$^\circ$C?
  
ou
  
P(temperatura máxima de hoje) = 32.354321$^\circ$C?

\vfill

Os dois tem a mesma probabilidade, que é \textbf{zero}.


$P(n_i) \approx F = \frac{n_i}{N} = \frac{1}{\inf} = 0$

  
$ \lim_{N \to \inf} P(x) = 0$ 

\end{frame} 
%===============================================================================%



%===============================================================================
\begin{frame}{Distribuições Contínuas}


Para V.A. contínuas, ao invés de massas de probabilidade, falamos de \textbf{densidades de probabilidade}, dentro de um intervalo de valores.

\vfill

As distribuições de probabilidade contínuas tem, desta maneira, \textbf{funções de densidade de probabilidade (f.d.p ou \emph{p.d.f})}

\vfill

Qual a probabilidade da temperatura máxima de hoje estar entre 32 e 33$^\circ$C?

Qual a probabilidade da temperatura máxima de hoje ser maior que 32$^\circ$C?
    
\end{frame} 
%===============================================================================%

%===============================================================================
\begin{frame}{Distribuições Contínuas}

Qual a distribuição contínua mais utilizada? \pause

\vfill

\textbf{Distribuição Normal (Gaussiana): $X \sim N(\mu,\sigma)$}

\begin{footnotesize}

Parâmetros: $\mu$ ($\mu \in \mathbb{R})$); $\sigma$ ($\sigma > 0$)

Suporte: $X \in \mathbb{R}$

p.d.f:

\begin{equation*}
P(x) = \frac{1}{{\sigma \sqrt {2\pi } }}e^{{{ - \left( {x - \mu } \right)^2 } \mathord{\left/ {\vphantom {{ - \left( {x - \mu } \right)^2 } {2\sigma ^2 }}} \right. \kern-\nulldelimiterspace} {2\sigma ^2 }}}
\end{equation*}

$E(X) = \mu$

$Var(X) = \sigma^2$ (Muitas vezes usamos o desvio padrão: $\sqrt{\sigma^2} = \sigma$)



\alert{\textbf{Exemplo:}} Qual a probabilidade de observamos uma temperatura entre 30$^\circ$C e 35$^\circ$C, se a média histórica é $\mu = 30$, e o desvio padrão é $\sigma = 5$?

\end{footnotesize}
    
\end{frame} 
%===============================================================================%

%===============================================================================
\begin{frame}[fragile]{Distribuição Normal - Exemplo}



\end{frame} 
%===============================================================================%


%===============================================================================
\begin{frame}[fragile]{Distribuição Normal - Exemplo}


\begin{Schunk}
\begin{Sinput}
> # No R
> media <- 30
> desvio <- 5
> tmin <- 30
> tmax <- 35
> # A maneira mais fácil de calcular uma probabilidade é de maneira cumulativa: P(x <= 30)
> #O comando "pnorm" faz isso:
> 
> p.tmin <- pnorm(tmin,mean=media,sd=desvio)
> p.tmin
\end{Sinput}
\begin{Soutput}
[1] 0.5
\end{Soutput}
\begin{Sinput}
> 
\end{Sinput}
\end{Schunk}

\end{frame} 
%===============================================================================%


%===============================================================================
\begin{frame}[fragile]{Distribuição Normal - Exemplo}


\end{frame} 
%===============================================================================%

%===============================================================================
\begin{frame}[fragile]{Distribuição Normal - Exemplo}


\begin{Schunk}
\begin{Sinput}
> # No R
> media <- 30
> desvio <- 5
> tmin <- 30
> tmax <- 35
> # calculamos também a probabilidade cumulativa de x <= 35
> 
> p.tmax <- pnorm(tmax,mean=media,sd=desvio)
> p.tmax
\end{Sinput}
\begin{Soutput}
[1] 0.8413447
\end{Soutput}
\begin{Sinput}
> 
\end{Sinput}
\end{Schunk}

\end{frame} 
%===============================================================================%


%===============================================================================
\begin{frame}[fragile]{Distribuição Normal - Exemplo}


\end{frame} 
%===============================================================================%

%===============================================================================
\begin{frame}[fragile]{Distribuição Normal - Exemplo}


\begin{Schunk}
\begin{Sinput}
> # Sabendo as duas probabilidades cumulativas, é só subtrair
> 
> p.tmax - p.tmin
\end{Sinput}
\begin{Soutput}
[1] 0.3413447
\end{Soutput}
\begin{Sinput}
> 
\end{Sinput}
\end{Schunk}

\end{frame} 
%===============================================================================%


%===============================================================================
\begin{frame}[fragile]{Distribuição Normal - Exemplo}


\end{frame} 
%===============================================================================%


%===============================================================================
\begin{frame}{Distribuições Contínuas}

\begin{small}
\begin{itemize}
  \item \textbf{Normal}: $X \sim N(\mu,\sigma)$
  \vfill
  \item \textbf{Gamma}: $X \sim Gamma(s,a$)
  \vfill
  \item \textbf{Exponencial}: $X \sim Exp(\lambda)$
  \vfill
  \item \textbf{Beta}: $X \sim Beta(a,b)$
  \vfill
  \item \textbf{Lognormal}: $X \sim logN(\mu,\sigma)$
  \vfill
  \item \textbf{Qui-quadrado}: $X \sim \chi^2(df)$
\end{itemize}
\end{small}

\end{frame} 
%===============================================================================%

\end{document}
